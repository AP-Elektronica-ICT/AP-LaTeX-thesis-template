\chapter{Opmerkingen bij scripties}
\label{chap_app1}
Overzicht van de opmerkingen bij het nalezen van scripties.

Deze opmerkingen later verwerken in de rest van de uitleg.

\section{Bibliografie}

\textit{Ik heb nog een vraag over het refereren. Is het de bedoeling dat de referentie in het begin of op het einde van de tekst (alinea) staat?}

Geen \'e\'en van de twee, het moet op de juiste plek staan.
Plaats verwijzingen naar de bibliografie op de juiste plek, dus niet steeds op het einde van een alinea.

Mooi voorbeeld van uit de scriptie van 2012:

De WRAP 920AR is in de verkoop gegaan in januari 2010, maar hij is nog steeds \'e\'en van de beste 
die er te koop is [1]. 
Deze bril kan als een gewone bril opgezet worden. Er komen wel meteen enkele praktische
nadelen bij kijken [2]. Ten eerste steken linkeroor 2 kabels naar buiten, waardoor de bril scheef
komt te staan op je hoofd. De kabels hebben de volgende functie: de eerste kabel stuurt de
webcam aan en is overbodig als de gebruiker alleen de displays wilt gebruiken. De tweede kabel
is verbonden met een afstandsbediening die dan op zijn beurt verbonden is met een USB-poort
en een VGA-poort. De USB-verbinding is nodig om de sensoren uit te lezen en de bril te
voorzien van stroom. De VGA-poort stuurt de 2 displays aan die in Windows herkend worden
als 1 extern scherm [3]. Het tweede nadeel is dat de elektronica dicht tegen het voorhoofd van de
gebruiker zit wordt, wat na een tijd begint te irriteren door de warmte ontwikkeling. [4]

Bij iedere referentie verwacht je als lezer iets anders, het is belangrijk dat je als schrijver uw referenties op de juiste plek zet.

\begin{itemize}
 \item $[1]$: info over de huidige markt, prijzen en specificaties
 \item $[2]$: artikel over de nadelen
 \item $[3]$: herkenning van bril in Windows
 \item $[4]$: hier verwacht ik een referentie over irraties
\end{itemize}



\section{Description}
Maak gebruik van ``description{}'' als je iets wil omschrijven (ipv itemize of enumerate)

\section{Etcetera}
Zet geen ``...'' of ``etc'' in uw tekst, dit doe je enkel in een chat of als je te lui bent om de volledige lijst op te zoeken

\url{http://grammarist.com/usage/et-cetera-etc}


\section{Zekerheden en twijfel}

Deze versie zou moeten werken onder Android en iOS.

Zeg ofwel: volgens de datasheet... volgens de specificatie...

Ofwel test iets voor je het zegt of meet het na


\section{Begin van een alinea}
``Voor de juiste positiebepaling van de marker zal ik eerst uitleggen hoe een camera (en oog) een 3D-wereld ziet.''
Vanop een bepaald punt (0,0,0) kijkt de camera in de diepte. Hoe verder er in de diepte wordt
gekeken hoe meer ook rechts en links alles zichtbaar wordt...

Zeg niet ``ik ga dit doen'', ``ik zal dat uitleggen'', leg het gewoon uit. Die eerste zin kan dus volledig geschrapt worden.

Een inleiding in het begin van een hoofdstuk om de structuur van het hoofdstuk te schetsen is uiteraard wel toegestaan.


\section{Broncode invoegen}

Dit kan je relatief proper met de listing package.

Voorbeeld:

\lstset{numbers=left, stepnumber=1, basicstyle=\footnotesize, language=C++, caption=AcceleroDice: Minimal Implementation, label=AcceleroDice, frame=none, xleftmargin=.3in}
\lstinputlisting{chapters/dice2.pde}






