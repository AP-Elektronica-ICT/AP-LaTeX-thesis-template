%**********************************************************************
%* Niets toevoegen aan dit bestand zonder toestemming van uw promotor *
%**********************************************************************
\documentclass[12pt,a4paper,oneside]{book}

%%%%%%%%%%%%%%%%%%%%%%%%%%%%%%%%%%%%%%%%%%%%%%%%%%%%%%%%%%%%%%%%%%%%%%%%%%%%%%%%%%
%% Volgende regel activeren voor eReader output:
%\usepackage[papersize={130mm,180mm},margin=2mm]{geometry}

%% Volgende regel activeren voor A4 output (bovenstaande terug afzetten)
\usepackage{a4wide} % Iets meer tekst op een bladzijde
%%%%%%%%%%%%%%%%%%%%%%%%%%%%%%%%%%%%%%%%%%%%%%%%%%%%%%%%%%%%%%%%%%%%%%%%%%%%%%%%%%

%% Packages %%
\usepackage{a4wide} 			% Iets meer tekst op een bladzijde
\usepackage[dutch]{babel} 		% Voor nederlandstalige woordsplitsing (en chapter hoofdingen)
\usepackage{amsmath} 			% Uitgebreide wiskundige mogelijkheden
\usepackage{url} 			% Om urls te verwerken
\usepackage{graphicx} 			% Om figuren te kunnen verwerken
\usepackage[latin1]{inputenc} 		% Om niet ascii karakters te kunnen typen
\usepackage[pagebackref=true]{hyperref}
\usepackage{listings}			% improve the code environments
%\usepackage{lmodern}
%\usepackage[T1]{fontenc}
\usepackage{textcomp}
\usepackage{pdfpages}			% Om pdf documenten in te kunnen voegen (titelpagina)
\usepackage{lipsum}			% Om lorem ipsum in te voegen
\usepackage{sty/eukdate}		% Om data in te voegen
\usepackage[DoggensLenny]{sty/fncychap}	% Added for nicer formatting: (kaders rond titels)
\usepackage{fix-cm}
\usepackage{parskip}			% Geen inspringing aan het begin van een alinea
%\usepackage{sectsty}
\usepackage{mdwlist}
\usepackage{sty/eukdate}		% Om data in te voegen
\usepackage{caption}
\usepackage{float}			% Om afbeelding geforceerd op een plek te plaatsten met 'H' parameter
\restylefloat{figure}
\usepackage{multirow}			% Weergave van complexere tabellen
\usepackage[scaled]{sty/inconsolata}	% Instelling lettertype
\usepackage{verbatim}

\hypersetup{				%% Markup for links in text, final versie enkel met zwarte links!!  %%
    colorlinks = false,  		%false=rode kaders rond tekst, true=links in de tekst zelf in kleur
    linkcolor = black,
    anchorcolor = black,
    citecolor = blue,
    filecolor = black,
    urlcolor = black
}

\renewcommand*\familydefault{\sfdefault} 			% Only if the base font of the document is to be sans serif
%\allsectionsfont{\usefont{OT1}{phv}{bc}{n}\selectfont}
  
\hyphenpenalty=5000						% Minder snel woorden splitsen
  \tolerance=1000		

%% Custom commands %%
\newcommand{\npar}{\par \vspace{2.3ex plus 0.3ex minus 0.3ex}}   % Enkele mm vertikale ruimte openlaten
\newcommand{\ntpar}{\par \vspace{1.3ex plus 0.3ex minus 0.3ex}}
\newcommand{\nhpar}{\par \vspace{4.3ex plus 0.3ex minus 0.3ex}}

\newcommand{\HRule}{\rule{\linewidth}{0.5mm}}			% Horizontale lijn

\setcounter{tocdepth}{2}					% weergeven tot subsection in table of contents

%\input{chapters/woordenlijst}					% Woordenlijst invoegen (nog testen)

%\usepackage[final]{sty/TODO}  			%TODO package
%\usepackage[silent]{sty/TODO}  			%TODO package 
\usepackage{sty/TODO}  			%TODO package

%*************************************************************************************
%* Plaats eventuele extra toegevoegde packages hier (na toestemming van uw promotor) *
%*************************************************************************************



