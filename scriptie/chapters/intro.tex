\pagenumbering{roman}

\selectlanguage{dutch}
\frontmatter
\thispagestyle{empty}   


\chapter{Dankwoord}
\vspace{0.35in}

Een voorwoord of dankwoord gebruik je om die mensen te bedanken zonder wie je de thesis niet had kunnen maken. 
Wie dat zijn, moet je zelf uitmaken, maar je promotor, respondenten en opdrachtgever horen in principe in dit lijstje thuis. 
De lengte van een voorwoord of dankwoord varieert meestal van een zestal regels tot een halve pagina.


\begin{flushright}{\emph{Antwerpen, \today \\
Voornaam Naam}}
\end{flushright}


\chapter{Abstract}
Het abstract geeft in een 200 tot 300 woorden de essentie van je thesis weer. 
Je schrijft kort en kernachtig je probleemstelling, methode, belangrijkste resultaten en conclusies neer. 
Je wacht dus best met het schrijven van je abstract tot je de conclusies duidelijk verwoord hebt.

Schenk aandacht aan de structuur van je abstract, houd rekening met het parallelle verwachtingspatroon. 
Als je in je methode zegt dat je de invloed van leeftijd en geslacht onderzocht hebt, houd deze volgorde dan ook aan in je resultaten. 

Denk eraan dat iemand vaak pas na het lezen van het abstract besluit om de rest van de tekst te lezen. 
Het spreekt daarom voor zich dat de abstract zelfstandig leesbaar moet zijn.

Alles wat overbodig en niet relevant is, moet uit je abstract blijven!

Op het internet kan je vele bronnen met informatie over de opbouw van een abstract vinden.
Specifiek voor onze opleiding geeft onderstaande website een goede uitleg.


\paragraph{Kort samengevat:}

\textit{
Despite the fact that an abstract is quite brief, it must do almost as much work as the multi-page paper that follows it. 
In a computer architecture paper, this means that it should in most cases include the following sections.
Each section is typically a single sentence, although there is room for creativity. 
In particular, the parts may be merged or spread among a set of sentences. Use the following as a checklist for your next abstract: 
}

\begin{itemize}
 \item motivation
 \item problem
 \item approach
 \item results
 \item conclusion
\end{itemize}

\url{http://www.ece.cmu.edu/~koopman/essays/abstract.html}]

% Inhoudstabel invoegen
\tableofcontents

% Lijst met alle tabellen invoegen
%\listoftables

% Lijst met alle figuren invoegen
\listoffigures